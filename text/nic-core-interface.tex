\subsection{The NIC Message Datapath}
\label{ssec:niccore-interface}
The \name{}'s NIC message datapath (Figure~\ref{fig:nanoPU}b) assembles arriving packets into a self-contained RPC message, then queues that message in a per-context FIFO until its destination thread is scheduled on a core by a nanokernel to be processed.
The message then flows, one word at a time, into the core's register file.
In the egress direction, messages sent by applications' threads are broken into packets before being sent to the NIC packet datapath.

We make two small modifications to the CPU core:
\begin{itemize}[topsep=0.4\baselineskip, leftmargin=20pt]
    \item Two general-purpose registers (GPRs) are now reserved for a special purpose: one is the \verb|HEAD| of the network receive queue, the other is the \verb|TAIL| of the network transmit queue.
    Applications must be compiled to avoid using reserved GPRs for temporary storage.
    Fortunately, \verb|gcc| makes it easy to reserve registers via command-line options~\cite{gcc-options}.
    \item New control status registers (CSRs) are added for out-of-band communication between the CPU and the NIC.
    These are used to configure the NIC with context IDs and to enable NIC-driven thread scheduling.
\end{itemize}

The NIC message datapath maintains transmit and receive FIFOs for each context (\ie, thread) that is currently pinned to each core.
It ensures that the currently running thread only sees (\ie, reads from and writes to) its own per-context FIFO via the \verb|HEAD| and \verb|TAIL| GPRs.

The NIC exposes a {\em message} interface to applications running on the core, instead of the more traditional packet interface.
This makes the \name{} NIC hardware responsible for the transport layer, which includes breaking messages into packets, ensuring reliable delivery across the network, and performing message reassembly at the receiving end.
To this end, messages between the NIC and core carry a small 8-byte header whose format is shown in Figure~\ref{fig:app-headers}.

\begin{figure}
  \centering
  \small
  \begin{minipage}[c]{0.9\linewidth}
  \lstinputlisting[language=riscv]{code/loopback.txt}
  \end{minipage}
  \vspace{-5pt}
  \captionof{lstlisting}{Loopback with increment: a simple RISC-V assembly program for the \name{} that waits for a \SI{16}{B} message to arrive, increments each word of the message, and returns it to the sender.}
  \label{lst:asm}
\end{figure}

\paragraph{A \name{} program (loopback-increment)}
To illustrate how software on the \name{} interacts with the NIC, Listing~\ref{lst:asm} shows a simple loopback-increment program in RISC-V assembly language.
The program continuously reads 16-byte messages (two 8-byte integers) from the network, increments the integers, and sends the messages back to their sender.
The program details are described below.

The \verb|entry| procedure registers a context ID and its priority with the NIC by first writing a value to both the \verb|lcurcontext| and \verb|lcurpriority| CSRs, then subsequently writing the value 1 to the \verb|lniccmd| CSR.
The \verb|lniccmd| CSR is a bit vector used by software to send commands to the NIC; in this case, it is used to tell the NIC to allocate an RX and TX queue for context ID $=0$ with priority level $=0$.
The \verb|lniccmd| CSR can also be used to remove context IDs or to update priority level.\footnote{Registering a context ID with the NIC is roughly equivalent to opening a socket on a modern OS.}

The \verb|wait_msg| procedure waits for a message to arrive in the RX queue by polling the \verb|lmsgsrdy| CSR until it is set by the NIC, indicating that the context has messages to process.
While it is waiting, the application tells the NIC that it is idle by setting the \verb|lidle| CSR during the polling loop.
The NIC thread scheduler uses the idle signal to evict waiting threads in order to schedule a new thread that has messages waiting to be processed.

The \verb|loopback_plus1_16B| procedure simply swaps the source and destination addresses by moving the context header (the first word of every message) from the head register (\verb|t5|) to the tail register (\verb|t6|), shown on line 19 (Listing~\ref{lst:asm}). 
It then increments every integer in the received message, appends them to the message being transmitted, and waits for the next message to arrive.

Applications that use variable-length messages can use the message length (in the context header) to read the correct number of words from the network RX queue.
If an application reads an empty RX queue, the resulting behavior is undefined---similar to reading an uninitialized variable.
