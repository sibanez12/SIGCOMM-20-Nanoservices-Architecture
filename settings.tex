%%
%% Configure the paper for preprint or camera-ready submission (but not both)
\def\setuppreprint{0}
\def\setupcameready{0}

%%
%% \BibTeX command to typeset BibTeX logo in the docs
%\AtBeginDocument{%
%  \providecommand\BibTeX{{%
%    \normalfont B\kern-0.5em{\scshape i\kern-0.25em b}\kern-0.8em\TeX}}}

%%
%% Rights management information.  This information is sent to you
%% when you complete the rights form.  These commands have SAMPLE
%% values in them; it is your responsibility as an author to replace
%% the commands and values with those provided to you when you
%% complete the rights form.
\if\setupcameready1
    \setcopyright{acmcopyright}
    \copyrightyear{2018}
    \acmYear{2018}
    \acmDOI{10.1145/1122445.1122456}
\else
    \setcopyright{none}
    \acmDOI{}
\fi

%%
%% These commands are for a PROCEEDINGS abstract or paper.
\if\setupcameready1
    \acmConference[Woodstock '18]{Woodstock '18: ACM Symposium on Neural
        Gaze Detection}{June 03--05, 2018}{Woodstock, NY}
    \acmBooktitle{Woodstock '18: ACM Symposium on Neural Gaze Detection,
        June 03--05, 2018, Woodstock, NY}
    \acmPrice{15.00}
    \acmISBN{978-1-4503-9999-9/18/06}
\else
    \acmISBN{}
\fi

%%
%% Submission ID.
%% Use this when submitting an article to a sponsored event. You'll
%% receive a unique submission ID from the organizers
%% of the event, and this ID should be used as the parameter to this command.
\if\setupcameready1
    \acmSubmissionID{123-A56-BU3}
\fi

%%
%% The majority of ACM publications use numbered citations and
%% references.  The command \citestyle{authoryear} switches to the
%% "author year" style.
%%
%% If you are preparing content for an event
%% sponsored by ACM SIGGRAPH, you must use the "author year" style of
%% citations and references.
%% Uncommenting
%% the next command will enable that style.
%\citestyle{acmauthoryear}

%% Set top matter
\if\setupcameready1
    \settopmatter{printacmref=true} % Reference format
    \settopmatter{printfolios=false} % Page numbers
\else
    \settopmatter{printacmref=false}
    \settopmatter{printfolios=true}
\fi

%%
%% Removes footnote with conference information in first column
\if\setupcameready0
    \renewcommand\footnotetextcopyrightpermission[1]{} 
\fi

%% Remove running page headers
\if\setupcameready1
    \def\removepageheaders{0} 
\else
    \def\removepageheaders{1}
\fi

%% Spacing
\setlength{\textfloatsep}{4pt}
%\setlength{\belowcaptionskip}{-4pt}

%%
%% Abbreviations
\newcommand{\ie}{{i.e.}}
\newcommand{\eg}{{e.g.}}
\newcommand{\ea}{{et al.}}
\newcommand{\name}{nanoPU\xspace}

%%
%% Comments
\if\setupcameready1
    \def\showcomments{0}
\else
    \if\setuppreprint1
        \def\showcomments{0}
    \else
        \def\showcomments{1}
    \fi
\fi
\newcommand{\xxx}[1]{\textcolor{red}{#1}}
\newcommand{\shahbaz}[1]{\todo[color=blue]{Shahbaz: #1}}
\newcommand{\nick}[1]{\todo[color=red]{Nick: #1}}
\newcommand{\steve}[1]{{\todo[color=purple]{Steve: #1}}}
\newcommand{\chang}[1]{\todo[color=olive]{Chang: #1}}
\newcommand{\alex}[1]{\todo[color=orange]{Alex: #1}}

%% Tables
\newcolumntype{M}[1]{>{\centering\arraybackslash}m{#1}}
%\newcolumntype{g}{>{\columncolor{Gray}}c}
%\newcolumntype{C}[1]{>{\centering}m{#1}}

%% Fonts
\definecolor{Gray}{gray}{0.9}
\newcommand{\cmark}{\ding{51}}
\newcommand{\xmark}{\ding{55}}

%% Hyphenations
\hyphenation{micro-second}
\hyphenation{time-scales}

%% Figures
\graphicspath{{figures/}}

%% Paragraphs
%\renewcommand{\paragraph}[1]{\\\indent{\textbf{\textsl{#1:}}}}

%% Others
\newcommand{\gplfronttext}{}

%%%%%%%%%%%%%%%%%%%%%%%%%%%%%%%%%
%% Listing for RISC-V Assembly %%
%%%%%%%%%%%%%%%%%%%%%%%%%%%%%%%%%
\usepackage{listings} % needed for the inclusion of source code

% the following is needed for syntax highlighting
\usepackage{color}

\definecolor{dkgreen}{rgb}{0,0.6,0}
\definecolor{gray}{rgb}{0.5,0.5,0.5}
\definecolor{mauve}{rgb}{0.58,0,0.82}

%\lstset{ %
%  language=C,       % the language of the code
%  basicstyle=\footnotesize,       % the size of the fonts that are used for the code
%  numbers=left,                   % where to put the line-numbers
%  numberstyle=\tiny\color{gray},  % the style that is used for the line-numbers
%  stepnumber=1,                   % the step between two line-numbers. If it's 1, each line 
%                                  % will be numbered
%  numbersep=5pt,                  % how far the line-numbers are from the code
%  backgroundcolor=\color{white},  % choose the background color. You must add \usepackage{color}
%  showspaces=false,               % show spaces adding particular underscores
%  showstringspaces=false,         % underline spaces within strings
%  showtabs=false,                 % show tabs within strings adding particular underscores
%  frame=single,                   % adds a frame around the code
%  rulecolor=\color{black},        % if not set, the frame-color may be changed on line-breaks within not-black text (e.g. commens (green here))
%  tabsize=4,                      % sets default tabsize to 2 spaces
%  captionpos=b,                   % sets the caption-position to bottom
%  breaklines=true,                % sets automatic line breaking
%  breakatwhitespace=false,        % sets if automatic breaks should only happen at whitespace
%  title=\lstname,                 % show the filename of files included with \lstinputlisting;
%                                  % also try caption instead of title
%  keywordstyle=\color{blue},          % keyword style
%  commentstyle=\color{dkgreen},       % comment style
%  stringstyle=\color{mauve},         % string literal style
%  escapeinside={\%*}{*)},            % if you want to add a comment within your code
%  morekeywords={*,...}               % if you want to add more keywords to the set
%}

% Commands for text
\newcommand{\ballnumber}[1]{\tikz[baseline=(myanchor.base)] \node[circle,fill=.,inner sep=1pt] (myanchor) {\color{-.}\bfseries\footnotesize #1};}