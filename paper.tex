\documentclass[sigconf]{acmart}

\usepackage[english]{babel}

\usepackage[textsize=tiny]{todonotes}

% Copyright
\renewcommand\footnotetextcopyrightpermission[1]{} % removes footnote with conference info
\setcopyright{none}
%\setcopyright{acmcopyright}
%\setcopyright{acmlicensed}
%\setcopyright{rightsretained}
%\setcopyright{usgov}
%\setcopyright{usgovmixed}
%\setcopyright{cagov}
%\setcopyright{cagovmixed}

\settopmatter{printacmref=false, printccs=false, printfolios=true}

% DOI
\acmDOI{}

% ISBN
\acmISBN{}

%Conference
%\acmConference[Submitted for review to SIGCOMM]{}
%\acmYear{2018}
%\copyrightyear{}

%% {} with no args suppresses printing of the price
\acmPrice{}


\begin{document}
\title{An Architecture for Nanoservices}

%\titlenote{Produces the permission block, and copyright information}
%\subtitle{Extended Abstract}

\author{Paper \# XXX, XXX pages}
% \author{Firstname Lastname}
% \authornote{Note}
% \orcid{1234-5678-9012}
% \affiliation{%
%   \institution{Affiliation}
%   \streetaddress{Address}
%   \city{City} 
%   \state{State} 
%   \postcode{Zipcode}
% }
% \email{email@domain.com}

% The default list of authors is too long for headers}
\renewcommand{\shortauthors}{X.et al.}

\begin{abstract}
Nanoservices are highly parallelizable, compute intensive applications with a cache resident working set and sub-microsecond response times. We believe that nanoservices will become popular in the future for two main reasons. One, Moore's Law is preventing increases in compute speed so we cannot rely on faster processors for performance gains. Two, many applications have inherent parallelizability that cannot be harnessed today because of large, unpredictable network delays and large overheads for small messages. In order to make nanoservices a reality, we need a new networking optimized compute platform. Thus, in this paper we present the Nanoservice Processing Unit (NanoPU) which has the following characteristics: (1) a fast path between the network and the core of the CPU to minimize average communication latency, (2) NIC driven thread scheduling to minimize tail response times, and (3) transport termination in the NIC to minimize communication overhead. We leave the third characteristic for future work. The NanoPU provides an order of magnitude lower average latency and XXX orders of magnitude lower tail latency than existing systems for nanoservice applications.
\end{abstract}

\maketitle

\section{Introduction}

\section{Nanoservices}
\todo[inline]{Motivate nanoservices.}
\todo[inline]{Describe what they are and how they fill today's needs.}

\section{NanoPU Design Overview}
\todo[inline]{Insert full system diagram.}

\subsection{NIC Datapath}
\todo[inline]{Describe arbiter, PISA pipeline, message reassembly buffer.}
\todo[inline]{Insert figure of the datapath.}

\subsection{NIC-Core Interface}
\todo[inline]{Describe per-active-context FIFO mechanism, clock-domain-crossing FIFO, packetization buffer. Describe how GPRs are reserved as well as new CSRs.}

\subsection{The NanoKernel and NIC-Driven Thread Scheduling}
\todo[inline]{Describe NanoKernel interaction with NIC. Describe nanoservice interaction with NanoKernel.}

\section{Programming the NanoPU with Nanoservice Applications}
\todo[inline]{Describe how nanoservices must be written on top of NanoPU and how it differs from traditional applications.}
\todo[inline]{Nanoservice apps should strive to process network data in FIFO order, should have minimal memory requirements, should be compose of single-threaded nanoservers that are directly communicating with one another.}

\section{Evaluations}

\subsection{Microbenchmarks}
\todo[inline]{Loopback latency, ingress/egress path latency, 64B pkt throughput measurements.}

\subsection{Bare Metal Application Evaluations}
\todo[inline]{This section shows the reduction in average latency as a result of the hardware fast path to the core of the CPU.}
\todo[inline]{Streaming (NFV style) app, NN inference node, Othello App, N-body simulation (???)}

\subsection{Thread Scheduling Evaluations}
\todo[inline]{This section shows the reduction in tail latency as a result of NIC-driven thread scheduling for nanoservice applications.}
\todo[inline]{Compare Linux-style scheduling on NanoPU to NIC-driven scheduling on NanoPU.}
\todo[inline]{Maybe also compare NIC-driven scheduling on NanoPU to Linux-style scheduling on IceNIC?}

\section{Related Work}

\section{Conclusion}




\bibliographystyle{ACM-Reference-Format}
\bibliography{reference}

\end{document}